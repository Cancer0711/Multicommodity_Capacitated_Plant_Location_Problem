\documentclass[a4paper,12pt,titlepage]{article}
\usepackage[utf8]{inputenc} 
\usepackage{tikz,pgf}
\usepackage{indentfirst}
\usepackage{amsfonts}
\usepackage[english]{babel}


%Url e Bookmarks of output PDF 
\usepackage{hyperref}
\hypersetup{
	colorlinks=true,
	linkcolor=blue,
	filecolor=magenta,      
	urlcolor=cyan,
%	pdftitle={Document title},
	bookmarks=true,
	%pdfpagemode=FullScreen,
}




\usepackage{rotating}

\usepackage{tabularx}
\usepackage{multirow} 
\usepackage{lscape}
\usepackage{tikz}
%to insert PDF files
\usepackage[final]{pdfpages}
%--Packages--

\usepackage{eurosym}
\usepackage{graphicx} \usepackage{verbatim}
\usepackage{graphics}
\usepackage{tikz,pgf}
\usepackage{indentfirst}
\usepackage{amsfonts}
\usepackage{graphicx}
\usepackage{amsmath}
\usepackage{amsmath,amssymb,amsthm,textcomp}
\usepackage{enumerate}
\usepackage{multicol}
\usepackage{tikz}
\usepackage{geometry}
\usepackage{mathtools}
\usepackage{amsmath}
\usepackage{verbatim}
\usepackage{amsmath,amssymb,mathrsfs}
\usepackage{xcolor}
\usepackage{graphicx,color,listings}
\frenchspacing 
\usepackage{geometry}
\usepackage{rotating}
\usepackage{caption}
\usepackage{xcolor}
\usepackage{listings}
%Cool maths printing
\usepackage{amsmath}
%PseudoCode
\usepackage{algorithm2e}


\begin{document}
\section*{Modello CPL (\textit{Capacitated Plant Location)} Multi prodotto}
\textbf{Insiemi}\\
- $V_1$ = insieme dei siti potenziali;\,\,(\textit{indice i});\\
\\
- $V_2$ = insieme dei siti da servire;\,\,(\textit{indice j});\\
\\
- $P$ = insieme dei prodotti;\,\,(\textit{indice p});\\
\\
- $A$ = insieme dei tragitti $(i,j)$ che uniscono i siti $i\in V_1$ ed $j\in V_2$\,;\\
\\
\textbf{Parametri}\\
- $d_{pj}$ = domanda di prodotto $p\in P$ richiesto da sito $j\in V_2$;\,\, $\forall p \in P$,\, $\forall j \in V_2$\\
\\
- $dist_{ij}$ = distanza tra sito $i\in V_1$ e sito $j\in V_2$;\,\,\,$\forall (i,j)\in A$\\
\\
- $c$ = costo di trasporto per unità di prodotto ed unità di distanza;\\
\textbf{\textit{Nota:}} Si suppone costante.\\
- $c_{pij}$ = costo per rifornire sito $j\in V_2$ tramite sito $i\in V_1$  soddisfacendo intera domanda $d_{pj}$; \,\, $\forall p \in P$,\,\,$\forall (i,j)\in A$\\
\begin{equation*}
	c_{pij} = 2\cdot c\cdot dist_{ij}\cdot d_{pj}
\end{equation*}
\textbf{\textit{Nota:}} Il valore 2 consente di tenere conto del costo associato sia all'andata che al ritorno, considerando di percorrere 2 volte la distanza $dist_{ij}$.\\
\\
\textbf{Variabili}\\
- $x_{pij}$ = frazione di domanda $d_{pj}$ espressa dal sito $j\in V_2$ e soddisfatta dal sito $i\in V_1$ relativamente al prodotto $p\in P$;\,\, $\forall p \in P$,\,\,$\forall (i,j)\in A$\\
\textbf{\textit{Nota:}} $0 \leq x_{pij}\leq 1$\\
\\
\[
-\,y_{i}=
\left\{
\begin{array}{ll}
1\quad\mbox{se il sito $i\in V_1$ è attivato};\,\,\forall i\in V_1 \qquad\qquad\qquad\qquad\qquad\qquad\qquad\qquad\qquad \\
0\quad\mbox{altrimenti}\\
\end{array}
\right.
\]



\newpage


\textbf{Condizione di Ammissibilità}\\
\begin{equation*}
	\sum_{p\in P}\sum_{j\in V_2} d_{pj} \leq \sum_{i\in V_1} q_i 
\end{equation*}
\\
\textbf{Formulazione Matematica}\\
\begin{equation}
	min\,\, \sum_{i\in V_1} f_i \cdot y_i + \sum_{p\in P}\sum_{j\in V_2} c_{pij} \cdot x_{pij}
	\tag{1}
\end{equation}
\begin{center}
	s.v
\end{center}
\begin{equation}
	\sum_{i\in V_1} x_{pij} = 1 \qquad \forall p \in P,\,\,\, \forall j \in V_2
	\tag{2}
\end{equation}

\begin{equation}
	\sum_{p\in P}\sum_{j\in V_2} d_{pj} \cdot x_{pij} \leq q_i \cdot y_i \qquad \forall i \in V_1
	\tag{3}
\end{equation}

\begin{equation}
	x_{pij}\geq 0 \qquad \forall p \in P,\,\,\,\forall (i,j)\in A
	\tag{4}
\end{equation}

\begin{equation}
	y_i \in \left\lbrace 0,1\right\rbrace  \qquad \forall i \in V_1
	\tag{5}
\end{equation}

\end{document}